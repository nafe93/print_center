
% Default to the notebook output style

    


% Inherit from the specified cell style.




    
\documentclass[11pt]{article}

    
    
    \usepackage[T1]{fontenc}
    % Nicer default font (+ math font) than Computer Modern for most use cases
    \usepackage{mathpazo}

    % Basic figure setup, for now with no caption control since it's done
    % automatically by Pandoc (which extracts ![](path) syntax from Markdown).
    \usepackage{graphicx}
    % We will generate all images so they have a width \maxwidth. This means
    % that they will get their normal width if they fit onto the page, but
    % are scaled down if they would overflow the margins.
    \makeatletter
    \def\maxwidth{\ifdim\Gin@nat@width>\linewidth\linewidth
    \else\Gin@nat@width\fi}
    \makeatother
    \let\Oldincludegraphics\includegraphics
    % Set max figure width to be 80% of text width, for now hardcoded.
    \renewcommand{\includegraphics}[1]{\Oldincludegraphics[width=.8\maxwidth]{#1}}
    % Ensure that by default, figures have no caption (until we provide a
    % proper Figure object with a Caption API and a way to capture that
    % in the conversion process - todo).
    \usepackage{caption}
    \DeclareCaptionLabelFormat{nolabel}{}
    \captionsetup{labelformat=nolabel}

    \usepackage{adjustbox} % Used to constrain images to a maximum size 
    \usepackage{xcolor} % Allow colors to be defined
    \usepackage{enumerate} % Needed for markdown enumerations to work
    \usepackage{geometry} % Used to adjust the document margins
    \usepackage{amsmath} % Equations
    \usepackage{amssymb} % Equations
    \usepackage{textcomp} % defines textquotesingle
    % Hack from http://tex.stackexchange.com/a/47451/13684:
    \AtBeginDocument{%
        \def\PYZsq{\textquotesingle}% Upright quotes in Pygmentized code
    }
    \usepackage{upquote} % Upright quotes for verbatim code
    \usepackage{eurosym} % defines \euro
    \usepackage[mathletters]{ucs} % Extended unicode (utf-8) support
    \usepackage[utf8x]{inputenc} % Allow utf-8 characters in the tex document
    \usepackage{fancyvrb} % verbatim replacement that allows latex
    \usepackage{grffile} % extends the file name processing of package graphics 
                         % to support a larger range 
    % The hyperref package gives us a pdf with properly built
    % internal navigation ('pdf bookmarks' for the table of contents,
    % internal cross-reference links, web links for URLs, etc.)
    \usepackage{hyperref}
    \usepackage{longtable} % longtable support required by pandoc >1.10
    \usepackage{booktabs}  % table support for pandoc > 1.12.2
    \usepackage[inline]{enumitem} % IRkernel/repr support (it uses the enumerate* environment)
    \usepackage[normalem]{ulem} % ulem is needed to support strikethroughs (\sout)
                                % normalem makes italics be italics, not underlines
    

    
    
    % Colors for the hyperref package
    \definecolor{urlcolor}{rgb}{0,.145,.698}
    \definecolor{linkcolor}{rgb}{.71,0.21,0.01}
    \definecolor{citecolor}{rgb}{.12,.54,.11}

    % ANSI colors
    \definecolor{ansi-black}{HTML}{3E424D}
    \definecolor{ansi-black-intense}{HTML}{282C36}
    \definecolor{ansi-red}{HTML}{E75C58}
    \definecolor{ansi-red-intense}{HTML}{B22B31}
    \definecolor{ansi-green}{HTML}{00A250}
    \definecolor{ansi-green-intense}{HTML}{007427}
    \definecolor{ansi-yellow}{HTML}{DDB62B}
    \definecolor{ansi-yellow-intense}{HTML}{B27D12}
    \definecolor{ansi-blue}{HTML}{208FFB}
    \definecolor{ansi-blue-intense}{HTML}{0065CA}
    \definecolor{ansi-magenta}{HTML}{D160C4}
    \definecolor{ansi-magenta-intense}{HTML}{A03196}
    \definecolor{ansi-cyan}{HTML}{60C6C8}
    \definecolor{ansi-cyan-intense}{HTML}{258F8F}
    \definecolor{ansi-white}{HTML}{C5C1B4}
    \definecolor{ansi-white-intense}{HTML}{A1A6B2}

    % commands and environments needed by pandoc snippets
    % extracted from the output of `pandoc -s`
    \providecommand{\tightlist}{%
      \setlength{\itemsep}{0pt}\setlength{\parskip}{0pt}}
    \DefineVerbatimEnvironment{Highlighting}{Verbatim}{commandchars=\\\{\}}
    % Add ',fontsize=\small' for more characters per line
    \newenvironment{Shaded}{}{}
    \newcommand{\KeywordTok}[1]{\textcolor[rgb]{0.00,0.44,0.13}{\textbf{{#1}}}}
    \newcommand{\DataTypeTok}[1]{\textcolor[rgb]{0.56,0.13,0.00}{{#1}}}
    \newcommand{\DecValTok}[1]{\textcolor[rgb]{0.25,0.63,0.44}{{#1}}}
    \newcommand{\BaseNTok}[1]{\textcolor[rgb]{0.25,0.63,0.44}{{#1}}}
    \newcommand{\FloatTok}[1]{\textcolor[rgb]{0.25,0.63,0.44}{{#1}}}
    \newcommand{\CharTok}[1]{\textcolor[rgb]{0.25,0.44,0.63}{{#1}}}
    \newcommand{\StringTok}[1]{\textcolor[rgb]{0.25,0.44,0.63}{{#1}}}
    \newcommand{\CommentTok}[1]{\textcolor[rgb]{0.38,0.63,0.69}{\textit{{#1}}}}
    \newcommand{\OtherTok}[1]{\textcolor[rgb]{0.00,0.44,0.13}{{#1}}}
    \newcommand{\AlertTok}[1]{\textcolor[rgb]{1.00,0.00,0.00}{\textbf{{#1}}}}
    \newcommand{\FunctionTok}[1]{\textcolor[rgb]{0.02,0.16,0.49}{{#1}}}
    \newcommand{\RegionMarkerTok}[1]{{#1}}
    \newcommand{\ErrorTok}[1]{\textcolor[rgb]{1.00,0.00,0.00}{\textbf{{#1}}}}
    \newcommand{\NormalTok}[1]{{#1}}
    
    % Additional commands for more recent versions of Pandoc
    \newcommand{\ConstantTok}[1]{\textcolor[rgb]{0.53,0.00,0.00}{{#1}}}
    \newcommand{\SpecialCharTok}[1]{\textcolor[rgb]{0.25,0.44,0.63}{{#1}}}
    \newcommand{\VerbatimStringTok}[1]{\textcolor[rgb]{0.25,0.44,0.63}{{#1}}}
    \newcommand{\SpecialStringTok}[1]{\textcolor[rgb]{0.73,0.40,0.53}{{#1}}}
    \newcommand{\ImportTok}[1]{{#1}}
    \newcommand{\DocumentationTok}[1]{\textcolor[rgb]{0.73,0.13,0.13}{\textit{{#1}}}}
    \newcommand{\AnnotationTok}[1]{\textcolor[rgb]{0.38,0.63,0.69}{\textbf{\textit{{#1}}}}}
    \newcommand{\CommentVarTok}[1]{\textcolor[rgb]{0.38,0.63,0.69}{\textbf{\textit{{#1}}}}}
    \newcommand{\VariableTok}[1]{\textcolor[rgb]{0.10,0.09,0.49}{{#1}}}
    \newcommand{\ControlFlowTok}[1]{\textcolor[rgb]{0.00,0.44,0.13}{\textbf{{#1}}}}
    \newcommand{\OperatorTok}[1]{\textcolor[rgb]{0.40,0.40,0.40}{{#1}}}
    \newcommand{\BuiltInTok}[1]{{#1}}
    \newcommand{\ExtensionTok}[1]{{#1}}
    \newcommand{\PreprocessorTok}[1]{\textcolor[rgb]{0.74,0.48,0.00}{{#1}}}
    \newcommand{\AttributeTok}[1]{\textcolor[rgb]{0.49,0.56,0.16}{{#1}}}
    \newcommand{\InformationTok}[1]{\textcolor[rgb]{0.38,0.63,0.69}{\textbf{\textit{{#1}}}}}
    \newcommand{\WarningTok}[1]{\textcolor[rgb]{0.38,0.63,0.69}{\textbf{\textit{{#1}}}}}
    
    
    % Define a nice break command that doesn't care if a line doesn't already
    % exist.
    \def\br{\hspace*{\fill} \\* }
    % Math Jax compatability definitions
    \def\gt{>}
    \def\lt{<}
    % Document parameters
    \title{center detection}
    
    
    

    % Pygments definitions
    
\makeatletter
\def\PY@reset{\let\PY@it=\relax \let\PY@bf=\relax%
    \let\PY@ul=\relax \let\PY@tc=\relax%
    \let\PY@bc=\relax \let\PY@ff=\relax}
\def\PY@tok#1{\csname PY@tok@#1\endcsname}
\def\PY@toks#1+{\ifx\relax#1\empty\else%
    \PY@tok{#1}\expandafter\PY@toks\fi}
\def\PY@do#1{\PY@bc{\PY@tc{\PY@ul{%
    \PY@it{\PY@bf{\PY@ff{#1}}}}}}}
\def\PY#1#2{\PY@reset\PY@toks#1+\relax+\PY@do{#2}}

\expandafter\def\csname PY@tok@w\endcsname{\def\PY@tc##1{\textcolor[rgb]{0.73,0.73,0.73}{##1}}}
\expandafter\def\csname PY@tok@c\endcsname{\let\PY@it=\textit\def\PY@tc##1{\textcolor[rgb]{0.25,0.50,0.50}{##1}}}
\expandafter\def\csname PY@tok@cp\endcsname{\def\PY@tc##1{\textcolor[rgb]{0.74,0.48,0.00}{##1}}}
\expandafter\def\csname PY@tok@k\endcsname{\let\PY@bf=\textbf\def\PY@tc##1{\textcolor[rgb]{0.00,0.50,0.00}{##1}}}
\expandafter\def\csname PY@tok@kp\endcsname{\def\PY@tc##1{\textcolor[rgb]{0.00,0.50,0.00}{##1}}}
\expandafter\def\csname PY@tok@kt\endcsname{\def\PY@tc##1{\textcolor[rgb]{0.69,0.00,0.25}{##1}}}
\expandafter\def\csname PY@tok@o\endcsname{\def\PY@tc##1{\textcolor[rgb]{0.40,0.40,0.40}{##1}}}
\expandafter\def\csname PY@tok@ow\endcsname{\let\PY@bf=\textbf\def\PY@tc##1{\textcolor[rgb]{0.67,0.13,1.00}{##1}}}
\expandafter\def\csname PY@tok@nb\endcsname{\def\PY@tc##1{\textcolor[rgb]{0.00,0.50,0.00}{##1}}}
\expandafter\def\csname PY@tok@nf\endcsname{\def\PY@tc##1{\textcolor[rgb]{0.00,0.00,1.00}{##1}}}
\expandafter\def\csname PY@tok@nc\endcsname{\let\PY@bf=\textbf\def\PY@tc##1{\textcolor[rgb]{0.00,0.00,1.00}{##1}}}
\expandafter\def\csname PY@tok@nn\endcsname{\let\PY@bf=\textbf\def\PY@tc##1{\textcolor[rgb]{0.00,0.00,1.00}{##1}}}
\expandafter\def\csname PY@tok@ne\endcsname{\let\PY@bf=\textbf\def\PY@tc##1{\textcolor[rgb]{0.82,0.25,0.23}{##1}}}
\expandafter\def\csname PY@tok@nv\endcsname{\def\PY@tc##1{\textcolor[rgb]{0.10,0.09,0.49}{##1}}}
\expandafter\def\csname PY@tok@no\endcsname{\def\PY@tc##1{\textcolor[rgb]{0.53,0.00,0.00}{##1}}}
\expandafter\def\csname PY@tok@nl\endcsname{\def\PY@tc##1{\textcolor[rgb]{0.63,0.63,0.00}{##1}}}
\expandafter\def\csname PY@tok@ni\endcsname{\let\PY@bf=\textbf\def\PY@tc##1{\textcolor[rgb]{0.60,0.60,0.60}{##1}}}
\expandafter\def\csname PY@tok@na\endcsname{\def\PY@tc##1{\textcolor[rgb]{0.49,0.56,0.16}{##1}}}
\expandafter\def\csname PY@tok@nt\endcsname{\let\PY@bf=\textbf\def\PY@tc##1{\textcolor[rgb]{0.00,0.50,0.00}{##1}}}
\expandafter\def\csname PY@tok@nd\endcsname{\def\PY@tc##1{\textcolor[rgb]{0.67,0.13,1.00}{##1}}}
\expandafter\def\csname PY@tok@s\endcsname{\def\PY@tc##1{\textcolor[rgb]{0.73,0.13,0.13}{##1}}}
\expandafter\def\csname PY@tok@sd\endcsname{\let\PY@it=\textit\def\PY@tc##1{\textcolor[rgb]{0.73,0.13,0.13}{##1}}}
\expandafter\def\csname PY@tok@si\endcsname{\let\PY@bf=\textbf\def\PY@tc##1{\textcolor[rgb]{0.73,0.40,0.53}{##1}}}
\expandafter\def\csname PY@tok@se\endcsname{\let\PY@bf=\textbf\def\PY@tc##1{\textcolor[rgb]{0.73,0.40,0.13}{##1}}}
\expandafter\def\csname PY@tok@sr\endcsname{\def\PY@tc##1{\textcolor[rgb]{0.73,0.40,0.53}{##1}}}
\expandafter\def\csname PY@tok@ss\endcsname{\def\PY@tc##1{\textcolor[rgb]{0.10,0.09,0.49}{##1}}}
\expandafter\def\csname PY@tok@sx\endcsname{\def\PY@tc##1{\textcolor[rgb]{0.00,0.50,0.00}{##1}}}
\expandafter\def\csname PY@tok@m\endcsname{\def\PY@tc##1{\textcolor[rgb]{0.40,0.40,0.40}{##1}}}
\expandafter\def\csname PY@tok@gh\endcsname{\let\PY@bf=\textbf\def\PY@tc##1{\textcolor[rgb]{0.00,0.00,0.50}{##1}}}
\expandafter\def\csname PY@tok@gu\endcsname{\let\PY@bf=\textbf\def\PY@tc##1{\textcolor[rgb]{0.50,0.00,0.50}{##1}}}
\expandafter\def\csname PY@tok@gd\endcsname{\def\PY@tc##1{\textcolor[rgb]{0.63,0.00,0.00}{##1}}}
\expandafter\def\csname PY@tok@gi\endcsname{\def\PY@tc##1{\textcolor[rgb]{0.00,0.63,0.00}{##1}}}
\expandafter\def\csname PY@tok@gr\endcsname{\def\PY@tc##1{\textcolor[rgb]{1.00,0.00,0.00}{##1}}}
\expandafter\def\csname PY@tok@ge\endcsname{\let\PY@it=\textit}
\expandafter\def\csname PY@tok@gs\endcsname{\let\PY@bf=\textbf}
\expandafter\def\csname PY@tok@gp\endcsname{\let\PY@bf=\textbf\def\PY@tc##1{\textcolor[rgb]{0.00,0.00,0.50}{##1}}}
\expandafter\def\csname PY@tok@go\endcsname{\def\PY@tc##1{\textcolor[rgb]{0.53,0.53,0.53}{##1}}}
\expandafter\def\csname PY@tok@gt\endcsname{\def\PY@tc##1{\textcolor[rgb]{0.00,0.27,0.87}{##1}}}
\expandafter\def\csname PY@tok@err\endcsname{\def\PY@bc##1{\setlength{\fboxsep}{0pt}\fcolorbox[rgb]{1.00,0.00,0.00}{1,1,1}{\strut ##1}}}
\expandafter\def\csname PY@tok@kc\endcsname{\let\PY@bf=\textbf\def\PY@tc##1{\textcolor[rgb]{0.00,0.50,0.00}{##1}}}
\expandafter\def\csname PY@tok@kd\endcsname{\let\PY@bf=\textbf\def\PY@tc##1{\textcolor[rgb]{0.00,0.50,0.00}{##1}}}
\expandafter\def\csname PY@tok@kn\endcsname{\let\PY@bf=\textbf\def\PY@tc##1{\textcolor[rgb]{0.00,0.50,0.00}{##1}}}
\expandafter\def\csname PY@tok@kr\endcsname{\let\PY@bf=\textbf\def\PY@tc##1{\textcolor[rgb]{0.00,0.50,0.00}{##1}}}
\expandafter\def\csname PY@tok@bp\endcsname{\def\PY@tc##1{\textcolor[rgb]{0.00,0.50,0.00}{##1}}}
\expandafter\def\csname PY@tok@fm\endcsname{\def\PY@tc##1{\textcolor[rgb]{0.00,0.00,1.00}{##1}}}
\expandafter\def\csname PY@tok@vc\endcsname{\def\PY@tc##1{\textcolor[rgb]{0.10,0.09,0.49}{##1}}}
\expandafter\def\csname PY@tok@vg\endcsname{\def\PY@tc##1{\textcolor[rgb]{0.10,0.09,0.49}{##1}}}
\expandafter\def\csname PY@tok@vi\endcsname{\def\PY@tc##1{\textcolor[rgb]{0.10,0.09,0.49}{##1}}}
\expandafter\def\csname PY@tok@vm\endcsname{\def\PY@tc##1{\textcolor[rgb]{0.10,0.09,0.49}{##1}}}
\expandafter\def\csname PY@tok@sa\endcsname{\def\PY@tc##1{\textcolor[rgb]{0.73,0.13,0.13}{##1}}}
\expandafter\def\csname PY@tok@sb\endcsname{\def\PY@tc##1{\textcolor[rgb]{0.73,0.13,0.13}{##1}}}
\expandafter\def\csname PY@tok@sc\endcsname{\def\PY@tc##1{\textcolor[rgb]{0.73,0.13,0.13}{##1}}}
\expandafter\def\csname PY@tok@dl\endcsname{\def\PY@tc##1{\textcolor[rgb]{0.73,0.13,0.13}{##1}}}
\expandafter\def\csname PY@tok@s2\endcsname{\def\PY@tc##1{\textcolor[rgb]{0.73,0.13,0.13}{##1}}}
\expandafter\def\csname PY@tok@sh\endcsname{\def\PY@tc##1{\textcolor[rgb]{0.73,0.13,0.13}{##1}}}
\expandafter\def\csname PY@tok@s1\endcsname{\def\PY@tc##1{\textcolor[rgb]{0.73,0.13,0.13}{##1}}}
\expandafter\def\csname PY@tok@mb\endcsname{\def\PY@tc##1{\textcolor[rgb]{0.40,0.40,0.40}{##1}}}
\expandafter\def\csname PY@tok@mf\endcsname{\def\PY@tc##1{\textcolor[rgb]{0.40,0.40,0.40}{##1}}}
\expandafter\def\csname PY@tok@mh\endcsname{\def\PY@tc##1{\textcolor[rgb]{0.40,0.40,0.40}{##1}}}
\expandafter\def\csname PY@tok@mi\endcsname{\def\PY@tc##1{\textcolor[rgb]{0.40,0.40,0.40}{##1}}}
\expandafter\def\csname PY@tok@il\endcsname{\def\PY@tc##1{\textcolor[rgb]{0.40,0.40,0.40}{##1}}}
\expandafter\def\csname PY@tok@mo\endcsname{\def\PY@tc##1{\textcolor[rgb]{0.40,0.40,0.40}{##1}}}
\expandafter\def\csname PY@tok@ch\endcsname{\let\PY@it=\textit\def\PY@tc##1{\textcolor[rgb]{0.25,0.50,0.50}{##1}}}
\expandafter\def\csname PY@tok@cm\endcsname{\let\PY@it=\textit\def\PY@tc##1{\textcolor[rgb]{0.25,0.50,0.50}{##1}}}
\expandafter\def\csname PY@tok@cpf\endcsname{\let\PY@it=\textit\def\PY@tc##1{\textcolor[rgb]{0.25,0.50,0.50}{##1}}}
\expandafter\def\csname PY@tok@c1\endcsname{\let\PY@it=\textit\def\PY@tc##1{\textcolor[rgb]{0.25,0.50,0.50}{##1}}}
\expandafter\def\csname PY@tok@cs\endcsname{\let\PY@it=\textit\def\PY@tc##1{\textcolor[rgb]{0.25,0.50,0.50}{##1}}}

\def\PYZbs{\char`\\}
\def\PYZus{\char`\_}
\def\PYZob{\char`\{}
\def\PYZcb{\char`\}}
\def\PYZca{\char`\^}
\def\PYZam{\char`\&}
\def\PYZlt{\char`\<}
\def\PYZgt{\char`\>}
\def\PYZsh{\char`\#}
\def\PYZpc{\char`\%}
\def\PYZdl{\char`\$}
\def\PYZhy{\char`\-}
\def\PYZsq{\char`\'}
\def\PYZdq{\char`\"}
\def\PYZti{\char`\~}
% for compatibility with earlier versions
\def\PYZat{@}
\def\PYZlb{[}
\def\PYZrb{]}
\makeatother


    % Exact colors from NB
    \definecolor{incolor}{rgb}{0.0, 0.0, 0.5}
    \definecolor{outcolor}{rgb}{0.545, 0.0, 0.0}



    
    % Prevent overflowing lines due to hard-to-break entities
    \sloppy 
    % Setup hyperref package
    \hypersetup{
      breaklinks=true,  % so long urls are correctly broken across lines
      colorlinks=true,
      urlcolor=urlcolor,
      linkcolor=linkcolor,
      citecolor=citecolor,
      }
    % Slightly bigger margins than the latex defaults
    
    \geometry{verbose,tmargin=1in,bmargin=1in,lmargin=1in,rmargin=1in}
    
    

    \begin{document}
    
    
    \maketitle
    
    

    
    \begin{Verbatim}[commandchars=\\\{\}]
{\color{incolor}In [{\color{incolor}3}]:} \PY{c+c1}{\PYZsh{}подключение библиотек}
        \PY{k+kn}{import} \PY{n+nn}{numpy} \PY{k}{as} \PY{n+nn}{np}
        \PY{k+kn}{import} \PY{n+nn}{cv2} \PY{k}{as} \PY{n+nn}{cv}
        \PY{k+kn}{from} \PY{n+nn}{math} \PY{k}{import} \PY{n}{pi}\PY{p}{,} \PY{n}{log10}\PY{p}{,} \PY{n}{sqrt}\PY{p}{,} \PY{n}{sin}\PY{p}{,} \PY{n}{cos}\PY{p}{,} \PY{n}{tan}\PY{p}{,} \PY{n}{atan}\PY{p}{,} \PY{n}{ceil}
        \PY{k+kn}{import} \PY{n+nn}{matplotlib}\PY{n+nn}{.}\PY{n+nn}{pyplot} \PY{k}{as} \PY{n+nn}{plt}
        
        \PY{o}{\PYZpc{}}\PY{k}{matplotlib} inline
        \PY{k+kn}{from} \PY{n+nn}{pylab} \PY{k}{import} \PY{n}{rcParams}
        \PY{n}{rcParams}\PY{p}{[}\PY{l+s+s1}{\PYZsq{}}\PY{l+s+s1}{figure.figsize}\PY{l+s+s1}{\PYZsq{}}\PY{p}{]} \PY{o}{=} \PY{l+m+mi}{20}\PY{p}{,} \PY{l+m+mi}{18}
\end{Verbatim}


    \subsection{Введение}\label{ux432ux432ux435ux434ux435ux43dux438ux435}

    Цель работы - найти центр круглой печати на изображении. Входными
данными является изображение, на котором присутствует круглая печать. В
постановке задачи неявно выдвинуто предположение, что на изображении
имеется объект, который является осесимметричным. Или, более того, на
изображении иеются достаточное количество концентрических окружностей. В
более общем случае вместо круглой печати может выступать любой объект,
содержащий в себе достаточное количество концентрических окружностей
(колесо, монета, радужка глаза, мяч).

Рассматривается два алгоритма нахождения центра объекта: основанный на
градиенте и основанный на структурном тензоре. Алгоритм нахождения
реперной точки следующий: 1. В каждой точке изображения на основе
градиента или структурного тензора определить вектор "направления". 2.
Создать "пустое" изображение того же размера, что и исходное 3. Через
каждую точку пустого изображения "прочертить" прямую в том направлении,
которое указано в соответствующей точке исходного изображения. Чтобы
нарисованные прямые не перекрывали друг друга, пиксели вдоль прямой
стоит не обновлять, а инкрементировать на определенную величину
(например 1) или на некоторый вес, определяемый каким-либо способом для
данной прямой 4. Каждый пиксель в вспомогательном изображении будет
равен сумме весов, проходящих через нее прямых. Таким образом, центром
объекта будет выступать аргмаксимум яркости вспомогательного изображения

    \subsection{Загрузка изображения и его
обработка}\label{ux437ux430ux433ux440ux443ux437ux43aux430-ux438ux437ux43eux431ux440ux430ux436ux435ux43dux438ux44f-ux438-ux435ux433ux43e-ux43eux431ux440ux430ux431ux43eux442ux43aux430}

    \begin{Verbatim}[commandchars=\\\{\}]
{\color{incolor}In [{\color{incolor}5}]:} \PY{c+c1}{\PYZsh{}load image}
        \PY{n}{img\PYZus{}stamp\PYZus{}color} \PY{o}{=} \PY{n}{cv}\PY{o}{.}\PY{n}{imread}\PY{p}{(}\PY{l+s+s1}{\PYZsq{}}\PY{l+s+s1}{stamp6.jpg}\PY{l+s+s1}{\PYZsq{}}\PY{p}{)}
        \PY{n}{img\PYZus{}stamp\PYZus{}color} \PY{o}{=} \PY{n}{cv}\PY{o}{.}\PY{n}{cvtColor}\PY{p}{(}\PY{n}{img\PYZus{}stamp\PYZus{}color}\PY{p}{,} \PY{n}{cv}\PY{o}{.}\PY{n}{COLOR\PYZus{}BGR2RGB}\PY{p}{)}
        \PY{n}{img\PYZus{}stamp} \PY{o}{=} \PY{n}{cv}\PY{o}{.}\PY{n}{cvtColor}\PY{p}{(}\PY{n}{img\PYZus{}stamp\PYZus{}color}\PY{p}{,} \PY{n}{cv}\PY{o}{.}\PY{n}{COLOR\PYZus{}RGB2GRAY}\PY{p}{)}
        \PY{n}{img\PYZus{}stamp} \PY{o}{=} \PY{n}{cv}\PY{o}{.}\PY{n}{bitwise\PYZus{}not}\PY{p}{(}\PY{n}{img\PYZus{}stamp}\PY{p}{)}
        
        \PY{c+c1}{\PYZsh{}cv.imshow(\PYZsq{}stamp\PYZsq{}, img\PYZus{}stamp)}
        \PY{c+c1}{\PYZsh{}k = cv.waitKey(0)}
        \PY{n}{plt}\PY{o}{.}\PY{n}{imshow}\PY{p}{(}\PY{n}{img\PYZus{}stamp}\PY{p}{,} \PY{n}{cmap}\PY{o}{=}\PY{l+s+s1}{\PYZsq{}}\PY{l+s+s1}{gray}\PY{l+s+s1}{\PYZsq{}}\PY{p}{)}\PY{p}{;}
\end{Verbatim}


    \begin{center}
    \adjustimage{max size={0.9\linewidth}{0.9\paperheight}}{output_4_0.png}
    \end{center}
    { \hspace*{\fill} \\}
    
    \begin{Verbatim}[commandchars=\\\{\}]
{\color{incolor}In [{\color{incolor}6}]:} \PY{c+c1}{\PYZsh{}image blur}
        \PY{n}{img\PYZus{}stamp\PYZus{}blur} \PY{o}{=} \PY{n}{cv}\PY{o}{.}\PY{n}{GaussianBlur}\PY{p}{(}\PY{n}{img\PYZus{}stamp}\PY{p}{,} \PY{n}{ksize}\PY{o}{=}\PY{p}{(}\PY{l+m+mi}{17}\PY{p}{,}\PY{l+m+mi}{17}\PY{p}{)}\PY{p}{,} \PY{n}{sigmaX}\PY{o}{=}\PY{l+m+mi}{5}\PY{p}{,} \PY{n}{sigmaY}\PY{o}{=}\PY{l+m+mi}{5}\PY{p}{)}
        
        \PY{c+c1}{\PYZsh{}cv.imshow(\PYZsq{}blur stamp\PYZsq{}, img\PYZus{}stamp\PYZus{}blur)}
        \PY{c+c1}{\PYZsh{}k = cv.waitKey(0)}
        \PY{n}{plt}\PY{o}{.}\PY{n}{imshow}\PY{p}{(}\PY{n}{img\PYZus{}stamp\PYZus{}blur}\PY{p}{,} \PY{n}{cmap}\PY{o}{=}\PY{l+s+s1}{\PYZsq{}}\PY{l+s+s1}{gray}\PY{l+s+s1}{\PYZsq{}}\PY{p}{)}\PY{p}{;}
\end{Verbatim}


    \begin{center}
    \adjustimage{max size={0.9\linewidth}{0.9\paperheight}}{output_5_0.png}
    \end{center}
    { \hspace*{\fill} \\}
    
    \begin{Verbatim}[commandchars=\\\{\}]
{\color{incolor}In [{\color{incolor}7}]:} \PY{c+c1}{\PYZsh{}image thresholding}
        \PY{c+c1}{\PYZsh{}window = 7}
        \PY{c+c1}{\PYZsh{}C = \PYZhy{}30}
        
        \PY{c+c1}{\PYZsh{}img\PYZus{}stamp\PYZus{}bin = cv.adaptiveThreshold(img\PYZus{}stamp, 255, cv.ADAPTIVE\PYZus{}THRESH\PYZus{}MEAN\PYZus{}C, cv.THRESH\PYZus{}BINARY, window, C)}
        \PY{n}{ret}\PY{p}{,} \PY{n}{img\PYZus{}stamp\PYZus{}blur\PYZus{}bin} \PY{o}{=} \PY{n}{cv}\PY{o}{.}\PY{n}{threshold}\PY{p}{(}\PY{n}{img\PYZus{}stamp\PYZus{}blur}\PY{p}{,} \PY{l+m+mi}{150}\PY{p}{,} \PY{l+m+mi}{255}\PY{p}{,} \PY{n}{cv}\PY{o}{.}\PY{n}{THRESH\PYZus{}TOZERO}\PY{p}{)}
        
        \PY{c+c1}{\PYZsh{}cv.imshow(\PYZsq{}adaptive thresholding stamp\PYZsq{}, img\PYZus{}stamp\PYZus{}blur\PYZus{}bin)}
        \PY{c+c1}{\PYZsh{}k = cv.waitKey(0)}
        \PY{n}{plt}\PY{o}{.}\PY{n}{imshow}\PY{p}{(}\PY{n}{img\PYZus{}stamp\PYZus{}blur\PYZus{}bin}\PY{p}{,} \PY{n}{cmap}\PY{o}{=}\PY{l+s+s1}{\PYZsq{}}\PY{l+s+s1}{gray}\PY{l+s+s1}{\PYZsq{}}\PY{p}{)}\PY{p}{;}
\end{Verbatim}


    \begin{center}
    \adjustimage{max size={0.9\linewidth}{0.9\paperheight}}{output_6_0.png}
    \end{center}
    { \hspace*{\fill} \\}
    
    \subsection{Дополнительные
функции}\label{ux434ux43eux43fux43eux43bux43dux438ux442ux435ux43bux44cux43dux44bux435-ux444ux443ux43dux43aux446ux438ux438}

    Напишем несколько дополнительных функций, которые будут необходимы в
дальнейшем. Как упоминалось ранее, пиксели вдоль прямой следует
инкрементировать на некоторую величину, а не обновлять значения в этих
пикселях.

    \subsection{Функиця инкремента значений вдоль
прямой}\label{ux444ux443ux43dux43aux438ux446ux44f-ux438ux43dux43aux440ux435ux43cux435ux43dux442ux430-ux437ux43dux430ux447ux435ux43dux438ux439-ux432ux434ux43eux43bux44c-ux43fux440ux44fux43cux43eux439}

    \begin{Verbatim}[commandchars=\\\{\}]
{\color{incolor}In [{\color{incolor}8}]:} \PY{c+c1}{\PYZsh{}вдоль отрезка от pt1 до pt2 увеличивает яркость пикселя на dcolor}
        \PY{c+c1}{\PYZsh{}тот же алгоритм Брезенхема, только инкрементирует значение в пикселе, а не выставляет новое}
        \PY{k}{def} \PY{n+nf}{straight}\PY{p}{(}\PY{n}{img}\PY{p}{,} \PY{n}{pt1}\PY{p}{,} \PY{n}{pt2}\PY{p}{,} \PY{n}{dcolor}\PY{p}{)}\PY{p}{:}
            \PY{n}{dx} \PY{o}{=} \PY{n}{pt2}\PY{p}{[}\PY{l+m+mi}{0}\PY{p}{]}\PY{o}{\PYZhy{}}\PY{n}{pt1}\PY{p}{[}\PY{l+m+mi}{0}\PY{p}{]}
            \PY{n}{dy} \PY{o}{=} \PY{n}{pt2}\PY{p}{[}\PY{l+m+mi}{1}\PY{p}{]}\PY{o}{\PYZhy{}}\PY{n}{pt1}\PY{p}{[}\PY{l+m+mi}{1}\PY{p}{]}
        
            \PY{n}{sign\PYZus{}x} \PY{o}{=} \PY{l+m+mi}{1} \PY{k}{if} \PY{n}{dx}\PY{o}{\PYZgt{}}\PY{l+m+mi}{0} \PY{k}{else} \PY{o}{\PYZhy{}}\PY{l+m+mi}{1} \PY{k}{if} \PY{n}{dx}\PY{o}{\PYZlt{}}\PY{l+m+mi}{0} \PY{k}{else} \PY{l+m+mi}{0}
            \PY{n}{sign\PYZus{}y} \PY{o}{=} \PY{l+m+mi}{1} \PY{k}{if} \PY{n}{dy}\PY{o}{\PYZgt{}}\PY{l+m+mi}{0} \PY{k}{else} \PY{o}{\PYZhy{}}\PY{l+m+mi}{1} \PY{k}{if} \PY{n}{dy}\PY{o}{\PYZlt{}}\PY{l+m+mi}{0} \PY{k}{else} \PY{l+m+mi}{0}
        
            \PY{k}{if} \PY{n}{dx} \PY{o}{\PYZlt{}} \PY{l+m+mi}{0}\PY{p}{:} \PY{n}{dx} \PY{o}{=} \PY{o}{\PYZhy{}}\PY{n}{dx}
            \PY{k}{if} \PY{n}{dy} \PY{o}{\PYZlt{}} \PY{l+m+mi}{0}\PY{p}{:} \PY{n}{dy} \PY{o}{=} \PY{o}{\PYZhy{}}\PY{n}{dy}
        
            \PY{k}{if} \PY{n}{dx} \PY{o}{\PYZgt{}} \PY{n}{dy}\PY{p}{:}
                \PY{n}{pdx}\PY{p}{,} \PY{n}{pdy} \PY{o}{=} \PY{n}{sign\PYZus{}x}\PY{p}{,} \PY{l+m+mi}{0}
                \PY{n}{es}\PY{p}{,} \PY{n}{el} \PY{o}{=} \PY{n}{dy}\PY{p}{,} \PY{n}{dx}
            \PY{k}{else}\PY{p}{:}
                \PY{n}{pdx}\PY{p}{,} \PY{n}{pdy} \PY{o}{=} \PY{l+m+mi}{0}\PY{p}{,} \PY{n}{sign\PYZus{}y}
                \PY{n}{es}\PY{p}{,} \PY{n}{el} \PY{o}{=} \PY{n}{dx}\PY{p}{,} \PY{n}{dy}
        
            \PY{n}{x}\PY{p}{,} \PY{n}{y} \PY{o}{=} \PY{n}{pt1}\PY{p}{[}\PY{l+m+mi}{0}\PY{p}{]}\PY{p}{,} \PY{n}{pt1}\PY{p}{[}\PY{l+m+mi}{1}\PY{p}{]}
        
            \PY{n}{error}\PY{p}{,} \PY{n}{t} \PY{o}{=} \PY{n}{el}\PY{o}{/}\PY{l+m+mi}{2}\PY{p}{,} \PY{l+m+mi}{0}        
        
            \PY{n}{img}\PY{p}{[}\PY{n}{x}\PY{p}{,}\PY{n}{y}\PY{p}{]} \PY{o}{+}\PY{o}{=} \PY{n}{dcolor}
        
            \PY{k}{while} \PY{n}{t} \PY{o}{\PYZlt{}} \PY{n}{el}\PY{p}{:}
                \PY{n}{error} \PY{o}{\PYZhy{}}\PY{o}{=} \PY{n}{es}
                \PY{k}{if} \PY{n}{error} \PY{o}{\PYZlt{}} \PY{l+m+mi}{0}\PY{p}{:}
                    \PY{n}{error} \PY{o}{+}\PY{o}{=} \PY{n}{el}
                    \PY{n}{x} \PY{o}{+}\PY{o}{=} \PY{n}{sign\PYZus{}x}
                    \PY{n}{y} \PY{o}{+}\PY{o}{=} \PY{n}{sign\PYZus{}y}
                \PY{k}{else}\PY{p}{:}
                    \PY{n}{x} \PY{o}{+}\PY{o}{=} \PY{n}{pdx}
                    \PY{n}{y} \PY{o}{+}\PY{o}{=} \PY{n}{pdy}
                \PY{n}{t} \PY{o}{+}\PY{o}{=} \PY{l+m+mi}{1}
                \PY{n}{img}\PY{p}{[}\PY{n}{x}\PY{p}{,}\PY{n}{y}\PY{p}{]} \PY{o}{+}\PY{o}{=} \PY{n}{dcolor}
\end{Verbatim}


    \subsection{Направление прямой на основе
градиента}\label{ux43dux430ux43fux440ux430ux432ux43bux435ux43dux438ux435-ux43fux440ux44fux43cux43eux439-ux43dux430-ux43eux441ux43dux43eux432ux435-ux433ux440ux430ux434ux438ux435ux43dux442ux430}

    Для нахождения направлений на исходном изображении так же напишем
отдельные функции. Первый способ - это нахождение направлений на основе
градиента яркости изображения. Функция находит градиент вдоль оси Ox и
оси Oy, используя оператор Собеля. После вычисляется угол наклона
градиента к оси Ox в каждой точке. Возвращается матрица со значениями
этих углов

    \begin{Verbatim}[commandchars=\\\{\}]
{\color{incolor}In [{\color{incolor}9}]:} \PY{c+c1}{\PYZsh{}направление на основе вектора градиента}
        \PY{k}{def} \PY{n+nf}{calcGV}\PY{p}{(}\PY{n}{inputIMG}\PY{p}{)}\PY{p}{:}
            \PY{n}{img} \PY{o}{=} \PY{n}{inputIMG}\PY{o}{.}\PY{n}{astype}\PY{p}{(}\PY{n}{np}\PY{o}{.}\PY{n}{float32}\PY{p}{)}
            \PY{c+c1}{\PYZsh{} GV components calculation (start)}
            \PY{n}{imgDiffX} \PY{o}{=} \PY{n}{cv}\PY{o}{.}\PY{n}{Sobel}\PY{p}{(}\PY{n}{img}\PY{p}{,} \PY{n}{cv}\PY{o}{.}\PY{n}{CV\PYZus{}32F}\PY{p}{,} \PY{l+m+mi}{1}\PY{p}{,} \PY{l+m+mi}{0}\PY{p}{,} \PY{l+m+mi}{3}\PY{p}{)}
            \PY{n}{imgDiffY} \PY{o}{=} \PY{n}{cv}\PY{o}{.}\PY{n}{Sobel}\PY{p}{(}\PY{n}{img}\PY{p}{,} \PY{n}{cv}\PY{o}{.}\PY{n}{CV\PYZus{}32F}\PY{p}{,} \PY{l+m+mi}{0}\PY{p}{,} \PY{l+m+mi}{1}\PY{p}{,} \PY{l+m+mi}{3}\PY{p}{)}
            \PY{c+c1}{\PYZsh{} GV components calculations (stop)}
            
            \PY{c+c1}{\PYZsh{} orientation angle calculation (start)}
            \PY{n}{imgOrientationOut} \PY{o}{=} \PY{n}{cv}\PY{o}{.}\PY{n}{phase}\PY{p}{(}\PY{n}{imgDiffX}\PY{p}{,} \PY{n}{imgDiffY}\PY{p}{,} \PY{n}{angleInDegrees} \PY{o}{=} \PY{k+kc}{False}\PY{p}{)}
            \PY{n}{imgOrientationOut} \PY{o}{=} \PY{n}{imgOrientationOut} \PY{o}{\PYZpc{}} \PY{n}{pi} \PY{c+c1}{\PYZsh{}[0,pi]}
            \PY{c+c1}{\PYZsh{} orientation angle calculation (stop)}
            
            \PY{k}{return} \PY{n}{imgOrientationOut}
\end{Verbatim}


    \subsection{Направление пямой на основе структурного
тензора}\label{ux43dux430ux43fux440ux430ux432ux43bux435ux43dux438ux435-ux43fux44fux43cux43eux439-ux43dux430-ux43eux441ux43dux43eux432ux435-ux441ux442ux440ux443ux43aux442ux443ux440ux43dux43eux433ux43e-ux442ux435ux43dux437ux43eux440ux430}

    Второй способ - это нахождение направлений на основе структурного
тензора Яне. Структурный тензор представляет из себя матрицу следующего
вида

    \(J = \begin{bmatrix}  J_{11} & J_{12} \\  J_{21} & J_{22}  \end{bmatrix}\)

    где \(J_{11} = M[Z_x^2]\), \(J_{22} = M[Z_y^2]\),
\(J_{12} = J_{21} = M[Z_xZ_y]\), \(M[]\) - математическое ожидание (в
нашем случае, например, среднее в гауссовом окне \(w\)), \(Z_x\),
\(Z_y\) - частные производные изображения по \(x\) и \(y\)

    Данная матрица вычисляется для каждой точки изображения. Она имеет
неотрицательные собственные значения. Собственный вектор,
соответствующий наибольшему собственному значению, будет направлен
перпендикулярно изменению градиента изображения. Угол наклона данного
вектора к оси Ox вычисляется следующим образом

    \$ \alpha = \frac{1}{2} \arctan \frac{2J_{12}}{J_{22}-J_{11}} \$

    и возвращается матрица со значениями посчитанных углов

    \begin{Verbatim}[commandchars=\\\{\}]
{\color{incolor}In [{\color{incolor}10}]:} \PY{c+c1}{\PYZsh{}направление на основе структурного тензора}
         \PY{k}{def} \PY{n+nf}{calcGST}\PY{p}{(}\PY{n}{inputIMG}\PY{p}{,} \PY{n}{w}\PY{p}{)}\PY{p}{:}
             \PY{n}{img} \PY{o}{=} \PY{n}{inputIMG}\PY{o}{.}\PY{n}{astype}\PY{p}{(}\PY{n}{np}\PY{o}{.}\PY{n}{float32}\PY{p}{)}
             \PY{c+c1}{\PYZsh{} GST components calculation (start)}
             \PY{c+c1}{\PYZsh{} J =  (J11 J12; J12 J22) \PYZhy{} GST}
             \PY{n}{imgDiffX} \PY{o}{=} \PY{n}{cv}\PY{o}{.}\PY{n}{Sobel}\PY{p}{(}\PY{n}{img}\PY{p}{,} \PY{n}{cv}\PY{o}{.}\PY{n}{CV\PYZus{}32F}\PY{p}{,} \PY{l+m+mi}{1}\PY{p}{,} \PY{l+m+mi}{0}\PY{p}{,} \PY{l+m+mi}{3}\PY{p}{)}
             \PY{n}{imgDiffY} \PY{o}{=} \PY{n}{cv}\PY{o}{.}\PY{n}{Sobel}\PY{p}{(}\PY{n}{img}\PY{p}{,} \PY{n}{cv}\PY{o}{.}\PY{n}{CV\PYZus{}32F}\PY{p}{,} \PY{l+m+mi}{0}\PY{p}{,} \PY{l+m+mi}{1}\PY{p}{,} \PY{l+m+mi}{3}\PY{p}{)}
             
             \PY{n}{imgDiffXY} \PY{o}{=} \PY{n}{cv}\PY{o}{.}\PY{n}{multiply}\PY{p}{(}\PY{n}{imgDiffX}\PY{p}{,} \PY{n}{imgDiffY}\PY{p}{)}
             \PY{n}{imgDiffXX} \PY{o}{=} \PY{n}{cv}\PY{o}{.}\PY{n}{multiply}\PY{p}{(}\PY{n}{imgDiffX}\PY{p}{,} \PY{n}{imgDiffX}\PY{p}{)}
             \PY{n}{imgDiffYY} \PY{o}{=} \PY{n}{cv}\PY{o}{.}\PY{n}{multiply}\PY{p}{(}\PY{n}{imgDiffY}\PY{p}{,} \PY{n}{imgDiffY}\PY{p}{)}
             
             \PY{n}{J11} \PY{o}{=} \PY{n}{cv}\PY{o}{.}\PY{n}{boxFilter}\PY{p}{(}\PY{n}{imgDiffXX}\PY{p}{,} \PY{n}{cv}\PY{o}{.}\PY{n}{CV\PYZus{}32F}\PY{p}{,} \PY{p}{(}\PY{n}{w}\PY{p}{,}\PY{n}{w}\PY{p}{)}\PY{p}{)}
             \PY{n}{J22} \PY{o}{=} \PY{n}{cv}\PY{o}{.}\PY{n}{boxFilter}\PY{p}{(}\PY{n}{imgDiffYY}\PY{p}{,} \PY{n}{cv}\PY{o}{.}\PY{n}{CV\PYZus{}32F}\PY{p}{,} \PY{p}{(}\PY{n}{w}\PY{p}{,}\PY{n}{w}\PY{p}{)}\PY{p}{)}
             \PY{n}{J12} \PY{o}{=} \PY{n}{cv}\PY{o}{.}\PY{n}{boxFilter}\PY{p}{(}\PY{n}{imgDiffXY}\PY{p}{,} \PY{n}{cv}\PY{o}{.}\PY{n}{CV\PYZus{}32F}\PY{p}{,} \PY{p}{(}\PY{n}{w}\PY{p}{,}\PY{n}{w}\PY{p}{)}\PY{p}{)}
             \PY{c+c1}{\PYZsh{} GST components calculations (stop)}
             
             \PY{c+c1}{\PYZsh{} eigenvalue calculation (start)}
             \PY{c+c1}{\PYZsh{} lambda1 = J11 + J22 + sqrt((J11\PYZhy{}J22)\PYZca{}2 + 4*J12\PYZca{}2)}
             \PY{c+c1}{\PYZsh{} lambda2 = J11 + J22 \PYZhy{} sqrt((J11\PYZhy{}J22)\PYZca{}2 + 4*J12\PYZca{}2)}
             \PY{n}{tmp1} \PY{o}{=} \PY{n}{J11} \PY{o}{+} \PY{n}{J22}
             \PY{n}{tmp2} \PY{o}{=} \PY{n}{J11} \PY{o}{\PYZhy{}} \PY{n}{J22}
             \PY{n}{tmp2} \PY{o}{=} \PY{n}{cv}\PY{o}{.}\PY{n}{multiply}\PY{p}{(}\PY{n}{tmp2}\PY{p}{,} \PY{n}{tmp2}\PY{p}{)}
             \PY{n}{tmp3} \PY{o}{=} \PY{n}{cv}\PY{o}{.}\PY{n}{multiply}\PY{p}{(}\PY{n}{J12}\PY{p}{,} \PY{n}{J12}\PY{p}{)}
             \PY{n}{tmp4} \PY{o}{=} \PY{n}{np}\PY{o}{.}\PY{n}{sqrt}\PY{p}{(}\PY{n}{tmp2} \PY{o}{+} \PY{l+m+mf}{4.0} \PY{o}{*} \PY{n}{tmp3}\PY{p}{)}
             \PY{n}{lambda1} \PY{o}{=} \PY{n}{tmp1} \PY{o}{+} \PY{n}{tmp4}    \PY{c+c1}{\PYZsh{} biggest eigenvalue}
             \PY{n}{lambda2} \PY{o}{=} \PY{n}{tmp1} \PY{o}{\PYZhy{}} \PY{n}{tmp4}    \PY{c+c1}{\PYZsh{} smallest eigenvalue}
             \PY{c+c1}{\PYZsh{} eigenvalue calculation (stop)}
             
             \PY{c+c1}{\PYZsh{} Coherency calculation (start)}
             \PY{c+c1}{\PYZsh{} Coherency = (lambda1 \PYZhy{} lambda2)/(lambda1 + lambda2)) \PYZhy{} measure of anisotropism}
             \PY{c+c1}{\PYZsh{} Coherency is anisotropy degree (consistency of local orientation)}
             \PY{c+c1}{\PYZsh{}\PYZsh{}imgCoherencyOut = cv.divide(lambda1 \PYZhy{} lambda2, lambda1 + lambda2)}
             \PY{c+c1}{\PYZsh{} Coherency calculation (stop)}
             
             \PY{c+c1}{\PYZsh{} orientation angle calculation (start)}
             \PY{c+c1}{\PYZsh{} tan(2*Alpha) = 2*J12/(J22 \PYZhy{} J11)}
             \PY{c+c1}{\PYZsh{} Alpha = 0.5 atan2(2*J12/(J22 \PYZhy{} J11))}
             \PY{n}{imgOrientationOut} \PY{o}{=} \PY{n}{cv}\PY{o}{.}\PY{n}{phase}\PY{p}{(}\PY{n}{J22} \PY{o}{\PYZhy{}} \PY{n}{J11}\PY{p}{,} \PY{l+m+mf}{2.0}\PY{o}{*}\PY{n}{J12}\PY{p}{,} \PY{n}{angleInDegrees} \PY{o}{=} \PY{k+kc}{False}\PY{p}{)}
             \PY{n}{imgOrientationOut} \PY{o}{=} \PY{p}{(}\PY{n}{pi}\PY{o}{/}\PY{l+m+mi}{2} \PY{o}{\PYZhy{}} \PY{l+m+mf}{0.5}\PY{o}{*}\PY{n}{imgOrientationOut}\PY{p}{)} \PY{o}{\PYZpc{}} \PY{n}{pi} \PY{c+c1}{\PYZsh{}[0,pi]}
             \PY{c+c1}{\PYZsh{} orientation angle calculation (stop)}
             
             \PY{k}{return} \PY{n}{imgOrientationOut}
\end{Verbatim}


    \subsection{Функция прорисовки линий по матрице углов этих
линий}\label{ux444ux443ux43dux43aux446ux438ux44f-ux43fux440ux43eux440ux438ux441ux43eux432ux43aux438-ux43bux438ux43dux438ux439-ux43fux43e-ux43cux430ux442ux440ux438ux446ux435-ux443ux433ux43bux43eux432-ux44dux442ux438ux445-ux43bux438ux43dux438ux439}

    Создадим функцию, которая принимает матрицу значений углов в каждой
точке и "чертит" прямую на пустом изображении через эту точку с тем
наклоном, который "записан" в этой точке. Функция возвращает изображение
с нарисованными прямыми. Для ускорения работы можно "пробегать" не
каждую точку, а с некоторым шагом step.

    \begin{Verbatim}[commandchars=\\\{\}]
{\color{incolor}In [{\color{incolor}11}]:} \PY{c+c1}{\PYZsh{}attention! O(n\PYZca{}3)}
         
         \PY{c+c1}{\PYZsh{}рисуем линию вдоль прямой по матрице углов}
         \PY{k}{def} \PY{n+nf}{draw\PYZus{}lines}\PY{p}{(}\PY{n}{alpha\PYZus{}matrix}\PY{p}{,} \PY{n}{step}\PY{o}{=}\PY{l+m+mi}{1}\PY{p}{)}\PY{p}{:}
             \PY{n}{img\PYZus{}lines} \PY{o}{=} \PY{n}{np}\PY{o}{.}\PY{n}{zeros}\PY{p}{(}\PY{n}{alpha\PYZus{}matrix}\PY{o}{.}\PY{n}{shape}\PY{p}{)}
         
             \PY{n}{y\PYZus{}max}\PY{p}{,} \PY{n}{x\PYZus{}max} \PY{o}{=} \PY{n}{alpha\PYZus{}matrix}\PY{o}{.}\PY{n}{shape}
             \PY{n}{y\PYZus{}max}\PY{p}{,} \PY{n}{x\PYZus{}max} \PY{o}{=} \PY{n}{y\PYZus{}max}\PY{o}{\PYZhy{}}\PY{l+m+mi}{1}\PY{p}{,} \PY{n}{x\PYZus{}max}\PY{o}{\PYZhy{}}\PY{l+m+mi}{1}
         
             \PY{k}{for} \PY{n}{i} \PY{o+ow}{in} \PY{n+nb}{range}\PY{p}{(}\PY{l+m+mi}{0}\PY{p}{,}\PY{n}{y\PYZus{}max}\PY{o}{+}\PY{l+m+mi}{1}\PY{p}{,}\PY{n}{step}\PY{p}{)}\PY{p}{:} \PY{c+c1}{\PYZsh{}проходим по строкам}
                 \PY{k}{for} \PY{n}{j} \PY{o+ow}{in} \PY{n+nb}{range}\PY{p}{(}\PY{l+m+mi}{0}\PY{p}{,}\PY{n}{x\PYZus{}max}\PY{o}{+}\PY{l+m+mi}{1}\PY{p}{,}\PY{n}{step}\PY{p}{)}\PY{p}{:} \PY{c+c1}{\PYZsh{}проходим по столбцам}
                     \PY{n}{x0} \PY{o}{=} \PY{n}{j}
                     \PY{n}{y0} \PY{o}{=} \PY{n}{i}
                     \PY{n}{alpha} \PY{o}{=} \PY{n}{alpha\PYZus{}matrix}\PY{p}{[}\PY{n}{i}\PY{p}{,}\PY{n}{j}\PY{p}{]}
         
                     \PY{c+c1}{\PYZsh{}находим точку пересечения с осями (точки начала и конца отрезка, соединяющие края изображения)}
                     \PY{n}{eps} \PY{o}{=} \PY{l+m+mf}{1e\PYZhy{}2}
                     \PY{k}{if} \PY{n+nb}{abs}\PY{p}{(}\PY{n}{alpha}\PY{p}{)}\PY{o}{\PYZgt{}}\PY{n}{eps} \PY{o+ow}{and} \PY{n+nb}{abs}\PY{p}{(}\PY{n}{pi}\PY{o}{/}\PY{l+m+mi}{2}\PY{o}{\PYZhy{}}\PY{n}{alpha}\PY{p}{)}\PY{o}{\PYZgt{}}\PY{n}{eps} \PY{o+ow}{and} \PY{n+nb}{abs}\PY{p}{(}\PY{n}{pi}\PY{o}{\PYZhy{}}\PY{n}{alpha}\PY{p}{)}\PY{o}{\PYZgt{}}\PY{n}{eps} \PY{o+ow}{and} \PY{n+nb}{abs}\PY{p}{(}\PY{l+m+mi}{3}\PY{o}{*}\PY{n}{pi}\PY{o}{/}\PY{l+m+mi}{2}\PY{o}{\PYZhy{}}\PY{n}{alpha}\PY{p}{)}\PY{o}{\PYZgt{}}\PY{n}{eps}\PY{p}{:} \PY{c+c1}{\PYZsh{}чтобы не рисовать вериткальные и горизонтальные линии}
                         \PY{n}{tgalpha} \PY{o}{=} \PY{n}{tan}\PY{p}{(}\PY{n}{alpha}\PY{p}{)} \PY{c+c1}{\PYZsh{}направление вдоль линий, поэтому берем tan}
                         
                         \PY{n}{P1} \PY{o}{=} \PY{p}{(}\PY{l+m+mi}{0}\PY{p}{,} \PY{n+nb}{int}\PY{p}{(}\PY{n}{tgalpha}\PY{o}{*}\PY{p}{(}\PY{l+m+mi}{0}\PY{o}{\PYZhy{}}\PY{n}{x0}\PY{p}{)}\PY{o}{+}\PY{n}{y0}\PY{p}{)}\PY{p}{)}
                         \PY{n}{P2} \PY{o}{=} \PY{p}{(}\PY{n}{x\PYZus{}max}\PY{p}{,} \PY{n+nb}{int}\PY{p}{(}\PY{n}{tgalpha}\PY{o}{*}\PY{p}{(}\PY{n}{x\PYZus{}max}\PY{o}{\PYZhy{}}\PY{n}{x0}\PY{p}{)}\PY{o}{+}\PY{n}{y0}\PY{p}{)}\PY{p}{)}
                         \PY{n}{P3} \PY{o}{=} \PY{p}{(}\PY{n+nb}{int}\PY{p}{(}\PY{l+m+mi}{1}\PY{o}{/}\PY{n}{tgalpha}\PY{o}{*}\PY{p}{(}\PY{l+m+mi}{0}\PY{o}{\PYZhy{}}\PY{n}{y0}\PY{p}{)}\PY{o}{+}\PY{n}{x0}\PY{p}{)}\PY{p}{,} \PY{l+m+mi}{0}\PY{p}{)}
                         \PY{n}{P4} \PY{o}{=} \PY{p}{(}\PY{n+nb}{int}\PY{p}{(}\PY{l+m+mi}{1}\PY{o}{/}\PY{n}{tgalpha}\PY{o}{*}\PY{p}{(}\PY{n}{y\PYZus{}max}\PY{o}{\PYZhy{}}\PY{n}{y0}\PY{p}{)}\PY{o}{+}\PY{n}{x0}\PY{p}{)}\PY{p}{,} \PY{n}{y\PYZus{}max}\PY{p}{)}
         
                         \PY{n}{points\PYZus{}cand} \PY{o}{=} \PY{p}{[}\PY{n}{P1}\PY{p}{,} \PY{n}{P2}\PY{p}{,} \PY{n}{P3}\PY{p}{,} \PY{n}{P4}\PY{p}{]}
                         \PY{n}{points\PYZus{}best} \PY{o}{=} \PY{p}{[}\PY{p}{]}
                         \PY{k}{for} \PY{n}{point} \PY{o+ow}{in} \PY{n}{points\PYZus{}cand}\PY{p}{:}
                             \PY{k}{if} \PY{l+m+mi}{0}\PY{o}{\PYZlt{}}\PY{o}{=}\PY{n}{point}\PY{p}{[}\PY{l+m+mi}{0}\PY{p}{]}\PY{o}{\PYZlt{}}\PY{o}{=}\PY{n}{x\PYZus{}max}\PY{p}{:}
                                 \PY{k}{if} \PY{l+m+mi}{0}\PY{o}{\PYZlt{}}\PY{o}{=}\PY{n}{point}\PY{p}{[}\PY{l+m+mi}{1}\PY{p}{]}\PY{o}{\PYZlt{}}\PY{o}{=}\PY{n}{y\PYZus{}max}\PY{p}{:}
                                     \PY{n}{points\PYZus{}best}\PY{o}{.}\PY{n}{append}\PY{p}{(}\PY{n}{point}\PY{p}{)}
         
                         \PY{n}{pt1}\PY{p}{,} \PY{o}{*}\PY{n}{gb}\PY{p}{,} \PY{n}{pt2} \PY{o}{=} \PY{n+nb}{sorted}\PY{p}{(}\PY{n}{points\PYZus{}best}\PY{p}{)}
                         
                         \PY{c+c1}{\PYZsh{}поменяем местами координату x(столбцы) и y(строки), т.к. функция line}
                         \PY{c+c1}{\PYZsh{}принимает точки с координатами (строка, столбец)}
                         \PY{n}{pt1} \PY{o}{=} \PY{p}{(}\PY{n}{pt1}\PY{p}{[}\PY{l+m+mi}{1}\PY{p}{]}\PY{p}{,} \PY{n}{pt1}\PY{p}{[}\PY{l+m+mi}{0}\PY{p}{]}\PY{p}{)}
                         \PY{n}{pt2} \PY{o}{=} \PY{p}{(}\PY{n}{pt2}\PY{p}{[}\PY{l+m+mi}{1}\PY{p}{]}\PY{p}{,} \PY{n}{pt2}\PY{p}{[}\PY{l+m+mi}{0}\PY{p}{]}\PY{p}{)}
         
                         \PY{c+c1}{\PYZsh{}рисуем линию по направлению производной}
                         \PY{n}{straight}\PY{p}{(}\PY{n}{img\PYZus{}lines}\PY{p}{,} \PY{n}{pt1}\PY{p}{,} \PY{n}{pt2}\PY{p}{,} \PY{l+m+mi}{1}\PY{p}{)}
         
             \PY{k}{return} \PY{n}{img\PYZus{}lines}
\end{Verbatim}


    \subsection{Проверка
алгоритма}\label{ux43fux440ux43eux432ux435ux440ux43aux430-ux430ux43bux433ux43eux440ux438ux442ux43cux430}

    \begin{Verbatim}[commandchars=\\\{\}]
{\color{incolor}In [{\color{incolor}12}]:} \PY{c+c1}{\PYZsh{}вычисляем направления на основе градиента и структурного тензора}
         \PY{n}{imgOrientation\PYZus{}GV} \PY{o}{=} \PY{n}{calcGV}\PY{p}{(}\PY{n}{img\PYZus{}stamp\PYZus{}blur}\PY{p}{)}
         \PY{n}{imgOrientation\PYZus{}GST} \PY{o}{=} \PY{n}{calcGST}\PY{p}{(}\PY{n}{img\PYZus{}stamp\PYZus{}blur}\PY{p}{,} \PY{l+m+mi}{7}\PY{p}{)}
\end{Verbatim}


    \begin{Verbatim}[commandchars=\\\{\}]
{\color{incolor}In [{\color{incolor}13}]:} \PY{o}{\PYZpc{}\PYZpc{}}\PY{k}{time}
         \PYZsh{}attention! O(n\PYZca{}3)
         
         \PYZsh{}\PYZdq{}чертим\PYZdq{} линии на основе градиента (матрица imgOrientation\PYZus{}GV)
         img\PYZus{}line\PYZus{}GV = draw\PYZus{}lines(imgOrientation\PYZus{}GV, 3)
\end{Verbatim}


    \begin{Verbatim}[commandchars=\\\{\}]
Wall time: 10.2 s

    \end{Verbatim}

    \begin{Verbatim}[commandchars=\\\{\}]
{\color{incolor}In [{\color{incolor}14}]:} \PY{c+c1}{\PYZsh{}полученное изображение \PYZdq{}начерченных\PYZdq{} линий на основе градиента}
         \PY{n}{img\PYZus{}line\PYZus{}GV} \PY{o}{=} \PY{n}{cv}\PY{o}{.}\PY{n}{normalize}\PY{p}{(}\PY{n}{img\PYZus{}line\PYZus{}GV}\PY{p}{,} \PY{k+kc}{None}\PY{p}{,} \PY{l+m+mi}{0}\PY{p}{,} \PY{l+m+mi}{255}\PY{p}{,} \PY{n}{cv}\PY{o}{.}\PY{n}{NORM\PYZus{}MINMAX}\PY{p}{,} \PY{n}{cv}\PY{o}{.}\PY{n}{CV\PYZus{}8U}\PY{p}{)}
         
         \PY{c+c1}{\PYZsh{}cv.imshow(\PYZsq{}line GV\PYZsq{}, img\PYZus{}line\PYZus{}GV)}
         \PY{c+c1}{\PYZsh{}k = cv.waitKey(0)}
         \PY{n}{plt}\PY{o}{.}\PY{n}{imshow}\PY{p}{(}\PY{n}{img\PYZus{}line\PYZus{}GV}\PY{p}{,} \PY{n}{cmap}\PY{o}{=}\PY{l+s+s1}{\PYZsq{}}\PY{l+s+s1}{gray}\PY{l+s+s1}{\PYZsq{}}\PY{p}{)}\PY{p}{;}
\end{Verbatim}


    \begin{center}
    \adjustimage{max size={0.9\linewidth}{0.9\paperheight}}{output_28_0.png}
    \end{center}
    { \hspace*{\fill} \\}
    
    \begin{Verbatim}[commandchars=\\\{\}]
{\color{incolor}In [{\color{incolor}15}]:} \PY{c+c1}{\PYZsh{}center}
         \PY{n}{i} \PY{o}{=} \PY{n}{img\PYZus{}line\PYZus{}GV}\PY{o}{.}\PY{n}{argmax}\PY{p}{(}\PY{p}{)} \PY{o}{/}\PY{o}{/} \PY{n}{img\PYZus{}line\PYZus{}GV}\PY{o}{.}\PY{n}{shape}\PY{p}{[}\PY{l+m+mi}{1}\PY{p}{]}
         \PY{n}{j} \PY{o}{=} \PY{n}{img\PYZus{}line\PYZus{}GV}\PY{o}{.}\PY{n}{argmax}\PY{p}{(}\PY{p}{)} \PY{o}{\PYZpc{}} \PY{n}{img\PYZus{}line\PYZus{}GV}\PY{o}{.}\PY{n}{shape}\PY{p}{[}\PY{l+m+mi}{1}\PY{p}{]}
         \PY{n+nb}{print}\PY{p}{(}\PY{l+s+s1}{\PYZsq{}}\PY{l+s+s1}{размер изображения:}\PY{l+s+s1}{\PYZsq{}}\PY{p}{,} \PY{n}{img\PYZus{}line\PYZus{}GV}\PY{o}{.}\PY{n}{shape}\PY{p}{)}
         \PY{n+nb}{print}\PY{p}{(}\PY{l+s+s1}{\PYZsq{}}\PY{l+s+s1}{центр объекта:}\PY{l+s+s1}{\PYZsq{}}\PY{p}{,} \PY{p}{(}\PY{n}{i}\PY{p}{,}\PY{n}{j}\PY{p}{)}\PY{p}{)}
\end{Verbatim}


    \begin{Verbatim}[commandchars=\\\{\}]
размер изображения: (439, 991)
центр объекта: (223, 355)

    \end{Verbatim}

    \begin{Verbatim}[commandchars=\\\{\}]
{\color{incolor}In [{\color{incolor}17}]:} \PY{c+c1}{\PYZsh{}найденный центр объекта на исходном изображении}
         \PY{n}{cv}\PY{o}{.}\PY{n}{circle}\PY{p}{(}\PY{n}{img\PYZus{}stamp\PYZus{}color}\PY{p}{,} \PY{p}{(}\PY{n}{j}\PY{p}{,}\PY{n}{i}\PY{p}{)}\PY{p}{,} \PY{l+m+mi}{10}\PY{p}{,} \PY{p}{(}\PY{l+m+mi}{0}\PY{p}{,}\PY{l+m+mi}{0}\PY{p}{,}\PY{l+m+mi}{255}\PY{p}{)}\PY{p}{,} \PY{l+m+mi}{2}\PY{p}{,} \PY{l+m+mi}{0}\PY{p}{)}
         \PY{n}{plt}\PY{o}{.}\PY{n}{imshow}\PY{p}{(}\PY{n}{img\PYZus{}stamp\PYZus{}color}\PY{p}{)}\PY{p}{;}
\end{Verbatim}


    \begin{center}
    \adjustimage{max size={0.9\linewidth}{0.9\paperheight}}{output_30_0.png}
    \end{center}
    { \hspace*{\fill} \\}
    
    \begin{Verbatim}[commandchars=\\\{\}]
{\color{incolor}In [{\color{incolor}18}]:} \PY{o}{\PYZpc{}\PYZpc{}}\PY{k}{time}
         \PYZsh{}attention! O(n\PYZca{}3)
         
         \PYZsh{}\PYZdq{}чертим\PYZdq{} линии на основе структурного тензора (матрица imgOrientation\PYZus{}GST)
         img\PYZus{}line\PYZus{}GST = draw\PYZus{}lines(imgOrientation\PYZus{}GST, 3)
\end{Verbatim}


    \begin{Verbatim}[commandchars=\\\{\}]
Wall time: 18.5 s

    \end{Verbatim}

    \begin{Verbatim}[commandchars=\\\{\}]
{\color{incolor}In [{\color{incolor}19}]:} \PY{c+c1}{\PYZsh{}полученное изображение \PYZdq{}начерченных\PYZdq{} линий на основе структурного тензора}
         \PY{n}{img\PYZus{}line\PYZus{}GST} \PY{o}{=} \PY{n}{cv}\PY{o}{.}\PY{n}{normalize}\PY{p}{(}\PY{n}{img\PYZus{}line\PYZus{}GST}\PY{p}{,} \PY{k+kc}{None}\PY{p}{,} \PY{l+m+mi}{0}\PY{p}{,} \PY{l+m+mi}{255}\PY{p}{,} \PY{n}{cv}\PY{o}{.}\PY{n}{NORM\PYZus{}MINMAX}\PY{p}{,} \PY{n}{cv}\PY{o}{.}\PY{n}{CV\PYZus{}8U}\PY{p}{)}
         
         \PY{c+c1}{\PYZsh{}cv.imshow(\PYZsq{}line GST\PYZsq{}, img\PYZus{}line\PYZus{}GST)}
         \PY{c+c1}{\PYZsh{}k = cv.waitKey(0)}
         \PY{n}{plt}\PY{o}{.}\PY{n}{imshow}\PY{p}{(}\PY{n}{img\PYZus{}line\PYZus{}GST}\PY{p}{,} \PY{n}{cmap}\PY{o}{=}\PY{l+s+s1}{\PYZsq{}}\PY{l+s+s1}{gray}\PY{l+s+s1}{\PYZsq{}}\PY{p}{)}\PY{p}{;}
\end{Verbatim}


    \begin{center}
    \adjustimage{max size={0.9\linewidth}{0.9\paperheight}}{output_32_0.png}
    \end{center}
    { \hspace*{\fill} \\}
    
    \begin{Verbatim}[commandchars=\\\{\}]
{\color{incolor}In [{\color{incolor}20}]:} \PY{c+c1}{\PYZsh{}center}
         \PY{n}{i} \PY{o}{=} \PY{n}{img\PYZus{}line\PYZus{}GST}\PY{o}{.}\PY{n}{argmax}\PY{p}{(}\PY{p}{)} \PY{o}{/}\PY{o}{/} \PY{n}{img\PYZus{}line\PYZus{}GST}\PY{o}{.}\PY{n}{shape}\PY{p}{[}\PY{l+m+mi}{1}\PY{p}{]}
         \PY{n}{j} \PY{o}{=} \PY{n}{img\PYZus{}line\PYZus{}GST}\PY{o}{.}\PY{n}{argmax}\PY{p}{(}\PY{p}{)} \PY{o}{\PYZpc{}} \PY{n}{img\PYZus{}line\PYZus{}GST}\PY{o}{.}\PY{n}{shape}\PY{p}{[}\PY{l+m+mi}{1}\PY{p}{]}
         \PY{n+nb}{print}\PY{p}{(}\PY{l+s+s1}{\PYZsq{}}\PY{l+s+s1}{размер изображения:}\PY{l+s+s1}{\PYZsq{}}\PY{p}{,} \PY{n}{img\PYZus{}line\PYZus{}GST}\PY{o}{.}\PY{n}{shape}\PY{p}{)}
         \PY{n+nb}{print}\PY{p}{(}\PY{l+s+s1}{\PYZsq{}}\PY{l+s+s1}{центр объекта:}\PY{l+s+s1}{\PYZsq{}}\PY{p}{,} \PY{p}{(}\PY{n}{i}\PY{p}{,}\PY{n}{j}\PY{p}{)}\PY{p}{)}
\end{Verbatim}


    \begin{Verbatim}[commandchars=\\\{\}]
размер изображения: (439, 991)
центр объекта: (222, 341)

    \end{Verbatim}

    \begin{Verbatim}[commandchars=\\\{\}]
{\color{incolor}In [{\color{incolor}21}]:} \PY{c+c1}{\PYZsh{}найденный центр объекта на исходном изображении}
         \PY{n}{cv}\PY{o}{.}\PY{n}{circle}\PY{p}{(}\PY{n}{img\PYZus{}stamp\PYZus{}color}\PY{p}{,} \PY{p}{(}\PY{n}{j}\PY{p}{,}\PY{n}{i}\PY{p}{)}\PY{p}{,} \PY{l+m+mi}{10}\PY{p}{,} \PY{p}{(}\PY{l+m+mi}{0}\PY{p}{,}\PY{l+m+mi}{255}\PY{p}{,}\PY{l+m+mi}{0}\PY{p}{)}\PY{p}{,} \PY{l+m+mi}{2}\PY{p}{,} \PY{l+m+mi}{0}\PY{p}{)}
         \PY{n}{plt}\PY{o}{.}\PY{n}{imshow}\PY{p}{(}\PY{n}{img\PYZus{}stamp\PYZus{}color}\PY{p}{)}\PY{p}{;}
\end{Verbatim}


    \begin{center}
    \adjustimage{max size={0.9\linewidth}{0.9\paperheight}}{output_34_0.png}
    \end{center}
    { \hspace*{\fill} \\}
    
    \subsection{Использование преобразования
Хафа}\label{ux438ux441ux43fux43eux43bux44cux437ux43eux432ux430ux43dux438ux435-ux43fux440ux435ux43eux431ux440ux430ux437ux43eux432ux430ux43dux438ux44f-ux445ux430ux444ux430}

    Алгоритм, основанный на "рисовании" линий имеет сложность \(O(n^3)\)
(\(n^2\) точек и для каждой точки "чертить" линию за \(O(n)\)). Его
можно ускорить, если использовать быстрое преобразование Хафа. Для этого
будем не "чертить" линию через точку, а находить параметры этой прямой в
\((s,t)\) параметризации. Преимущественно вертикальные и горизонтальные
прямые будем рассматривать отдельно. Понятно, что если наше изображение
имеет размер \((width, height)\), то для преимущественно горизонтальных
прямых параметр \(s \in [-width, height+width]\), а параметр
\(t\in[-width,width]\). Для преимущественно вертикальных прямых
\(s \in [-height, width+height]\), \(t\in[-height,height]\)

    \begin{figure}
\centering
\includegraphics{coef.jpg}
\caption{title}
\end{figure}

    Применим быстрое преобразование Хафа для пространства коэффициентов и
найдем аргмаксимум полученного образа. Данный аргмаксимум определяет в
пространстве коэффициентов наилучшую прямую, интеграл через который
максимален. Данная прямая в пространстве коэффициентов может быть задана
своими крайними точками \((s_1,t_1)\) и \((s_2,t_2)\). Она однозначно
определяет точку в исходном изображении, которая будет искомым центром
объекта.

Так как преимущественно горизонтальные и вертикальные прямые
рассматривались отдельно, то есть у нас два изображения с пространством
коэффициентов, то и в конечном итоге точки мы получим две, которые,
возможно, не совпадут. Чтобы этого избежать, немного модифицируем
процесс. Преимущественно горизонтальные прямые хорошо определяют центр
объекта по вертикали и наоборот, преимущественно вертикальные прямые
хорошо определяют центр объекта по горизонтали. Поэтому, в пространстве
коэффициентов найдем точку пересечения наилучшей прямой, которую мы
определили двумя точками \((s_1,t_1)\) и \((s_2,t_2)\) с прямой \(t=0\),
а вернее найдем абсциссу этой точки. Так как у нас два изображения
пространств коэффициентов, то и абсцисс мы получим две, а они как раз и
будут координатами центра нашего объекта с точностью до сдвига СК
пространств коэффициентов.

    \subsection{Функция заполнения пространства коэффициентов линий по
матрице углов этих
линий}\label{ux444ux443ux43dux43aux446ux438ux44f-ux437ux430ux43fux43eux43bux43dux435ux43dux438ux44f-ux43fux440ux43eux441ux442ux440ux430ux43dux441ux442ux432ux430-ux43aux43eux44dux444ux444ux438ux446ux438ux435ux43dux442ux43eux432-ux43bux438ux43dux438ux439-ux43fux43e-ux43cux430ux442ux440ux438ux446ux435-ux443ux433ux43bux43eux432-ux44dux442ux438ux445-ux43bux438ux43dux438ux439}

    \begin{Verbatim}[commandchars=\\\{\}]
{\color{incolor}In [{\color{incolor}22}]:} \PY{k}{def} \PY{n+nf}{calculate\PYZus{}Hough}\PY{p}{(}\PY{n}{alpha\PYZus{}matrix}\PY{p}{)}\PY{p}{:}
             \PY{n}{height}\PY{p}{,} \PY{n}{width} \PY{o}{=} \PY{n}{alpha\PYZus{}matrix}\PY{o}{.}\PY{n}{shape}
             \PY{n}{img\PYZus{}mostly\PYZus{}horizontal} \PY{o}{=} \PY{n}{np}\PY{o}{.}\PY{n}{zeros}\PY{p}{(}\PY{p}{(}\PY{l+m+mi}{2}\PY{o}{*}\PY{n}{width}\PY{p}{,} \PY{l+m+mi}{2}\PY{o}{*}\PY{n}{width}\PY{o}{+}\PY{n}{height}\PY{p}{)}\PY{p}{)}
             \PY{n}{img\PYZus{}mostly\PYZus{}vertical} \PY{o}{=} \PY{n}{np}\PY{o}{.}\PY{n}{zeros}\PY{p}{(}\PY{p}{(}\PY{l+m+mi}{2}\PY{o}{*}\PY{n}{height}\PY{p}{,} \PY{l+m+mi}{2}\PY{o}{*}\PY{n}{height}\PY{o}{+}\PY{n}{width}\PY{p}{)}\PY{p}{)}
         
             \PY{n}{y\PYZus{}max}\PY{p}{,} \PY{n}{x\PYZus{}max} \PY{o}{=} \PY{n}{alpha\PYZus{}matrix}\PY{o}{.}\PY{n}{shape}
             \PY{n}{y\PYZus{}max}\PY{p}{,} \PY{n}{x\PYZus{}max} \PY{o}{=} \PY{n}{y\PYZus{}max}\PY{o}{\PYZhy{}}\PY{l+m+mi}{1}\PY{p}{,} \PY{n}{x\PYZus{}max}\PY{o}{\PYZhy{}}\PY{l+m+mi}{1}
         
             \PY{k}{for} \PY{n}{i} \PY{o+ow}{in} \PY{n+nb}{range}\PY{p}{(}\PY{n}{y\PYZus{}max}\PY{o}{+}\PY{l+m+mi}{1}\PY{p}{)}\PY{p}{:} \PY{c+c1}{\PYZsh{}проходим по строкам}
                 \PY{k}{for} \PY{n}{j} \PY{o+ow}{in} \PY{n+nb}{range}\PY{p}{(}\PY{n}{x\PYZus{}max}\PY{o}{+}\PY{l+m+mi}{1}\PY{p}{)}\PY{p}{:} \PY{c+c1}{\PYZsh{}проходим по столбцам}
                     \PY{n}{x0} \PY{o}{=} \PY{n}{j}
                     \PY{n}{y0} \PY{o}{=} \PY{n}{i}
                     \PY{n}{alpha} \PY{o}{=} \PY{n}{alpha\PYZus{}matrix}\PY{p}{[}\PY{n}{i}\PY{p}{,}\PY{n}{j}\PY{p}{]}
         
                     \PY{c+c1}{\PYZsh{}находим точку пересечения с осями (точки начала и конца отрезка, соединяющие края изображения)}
                     \PY{n}{eps} \PY{o}{=} \PY{l+m+mf}{1e\PYZhy{}2}
                     \PY{k}{if} \PY{n+nb}{abs}\PY{p}{(}\PY{n}{alpha}\PY{p}{)}\PY{o}{\PYZgt{}}\PY{n}{eps} \PY{o+ow}{and} \PY{n+nb}{abs}\PY{p}{(}\PY{n}{pi}\PY{o}{/}\PY{l+m+mi}{2}\PY{o}{\PYZhy{}}\PY{n}{alpha}\PY{p}{)}\PY{o}{\PYZgt{}}\PY{n}{eps} \PY{o+ow}{and} \PY{n+nb}{abs}\PY{p}{(}\PY{n}{pi}\PY{o}{\PYZhy{}}\PY{n}{alpha}\PY{p}{)}\PY{o}{\PYZgt{}}\PY{n}{eps} \PY{o+ow}{and} \PY{n+nb}{abs}\PY{p}{(}\PY{l+m+mi}{3}\PY{o}{*}\PY{n}{pi}\PY{o}{/}\PY{l+m+mi}{2}\PY{o}{\PYZhy{}}\PY{n}{alpha}\PY{p}{)}\PY{o}{\PYZgt{}}\PY{n}{eps}\PY{p}{:} \PY{c+c1}{\PYZsh{}чтобы не рисовать вериткальные и горизонтальные линии}
                         \PY{n}{tgalpha} \PY{o}{=} \PY{n}{tan}\PY{p}{(}\PY{n}{alpha}\PY{p}{)} \PY{c+c1}{\PYZsh{}направление вдоль линий, поэтому берем tan}
                         
                         \PY{k}{if} \PY{n+nb}{abs}\PY{p}{(}\PY{n}{tgalpha}\PY{p}{)} \PY{o}{\PYZlt{}}\PY{o}{=} \PY{l+m+mi}{1}\PY{p}{:} \PY{c+c1}{\PYZsh{}т.е. преимущественно горизонтальная прямая}
                             \PY{n}{P1} \PY{o}{=} \PY{p}{(}\PY{l+m+mi}{0}\PY{p}{,} \PY{n+nb}{int}\PY{p}{(}\PY{n}{tgalpha}\PY{o}{*}\PY{p}{(}\PY{l+m+mi}{0}\PY{o}{\PYZhy{}}\PY{n}{x0}\PY{p}{)}\PY{o}{+}\PY{n}{y0}\PY{p}{)}\PY{p}{)}
                             \PY{n}{P2} \PY{o}{=} \PY{p}{(}\PY{n}{x\PYZus{}max}\PY{p}{,} \PY{n+nb}{int}\PY{p}{(}\PY{n}{tgalpha}\PY{o}{*}\PY{p}{(}\PY{n}{x\PYZus{}max}\PY{o}{\PYZhy{}}\PY{n}{x0}\PY{p}{)}\PY{o}{+}\PY{n}{y0}\PY{p}{)}\PY{p}{)}
                             
                             \PY{n}{s} \PY{o}{=} \PY{n}{P1}\PY{p}{[}\PY{l+m+mi}{1}\PY{p}{]} \PY{o}{+} \PY{n}{width}
                             \PY{n}{t} \PY{o}{=} \PY{n}{P2}\PY{p}{[}\PY{l+m+mi}{1}\PY{p}{]}\PY{o}{\PYZhy{}}\PY{n}{P1}\PY{p}{[}\PY{l+m+mi}{1}\PY{p}{]} \PY{o}{+} \PY{n}{width}
                             
                             \PY{n}{img\PYZus{}mostly\PYZus{}horizontal}\PY{p}{[}\PY{n}{t}\PY{p}{,}\PY{n}{s}\PY{p}{]} \PY{o}{+}\PY{o}{=} \PY{l+m+mi}{1}
                         \PY{k}{else}\PY{p}{:} \PY{c+c1}{\PYZsh{}т.е. преимущественно вертикальная прямая}
                             \PY{n}{P3} \PY{o}{=} \PY{p}{(}\PY{n+nb}{int}\PY{p}{(}\PY{l+m+mi}{1}\PY{o}{/}\PY{n}{tgalpha}\PY{o}{*}\PY{p}{(}\PY{l+m+mi}{0}\PY{o}{\PYZhy{}}\PY{n}{y0}\PY{p}{)}\PY{o}{+}\PY{n}{x0}\PY{p}{)}\PY{p}{,} \PY{l+m+mi}{0}\PY{p}{)}
                             \PY{n}{P4} \PY{o}{=} \PY{p}{(}\PY{n+nb}{int}\PY{p}{(}\PY{l+m+mi}{1}\PY{o}{/}\PY{n}{tgalpha}\PY{o}{*}\PY{p}{(}\PY{n}{y\PYZus{}max}\PY{o}{\PYZhy{}}\PY{n}{y0}\PY{p}{)}\PY{o}{+}\PY{n}{x0}\PY{p}{)}\PY{p}{,} \PY{n}{y\PYZus{}max}\PY{p}{)}
                             
                             \PY{n}{s} \PY{o}{=} \PY{n}{P3}\PY{p}{[}\PY{l+m+mi}{0}\PY{p}{]} \PY{o}{+} \PY{n}{height}
                             \PY{n}{t} \PY{o}{=} \PY{n}{P4}\PY{p}{[}\PY{l+m+mi}{0}\PY{p}{]}\PY{o}{\PYZhy{}}\PY{n}{P3}\PY{p}{[}\PY{l+m+mi}{0}\PY{p}{]} \PY{o}{+} \PY{n}{height}
                             
                             \PY{n}{img\PYZus{}mostly\PYZus{}vertical}\PY{p}{[}\PY{n}{t}\PY{p}{,}\PY{n}{s}\PY{p}{]} \PY{o}{+}\PY{o}{=} \PY{l+m+mi}{1}
         
             \PY{k}{return} \PY{n}{img\PYZus{}mostly\PYZus{}horizontal}\PY{p}{,} \PY{n}{img\PYZus{}mostly\PYZus{}vertical}
\end{Verbatim}


    \subsection{Функция нахождения координат центра объекта, используя
пространство
коэффициентов}\label{ux444ux443ux43dux43aux446ux438ux44f-ux43dux430ux445ux43eux436ux434ux435ux43dux438ux44f-ux43aux43eux43eux440ux434ux438ux43dux430ux442-ux446ux435ux43dux442ux440ux430-ux43eux431ux44aux435ux43aux442ux430-ux438ux441ux43fux43eux43bux44cux437ux443ux44f-ux43fux440ux43eux441ux442ux440ux430ux43dux441ux442ux432ux43e-ux43aux43eux44dux444ux444ux438ux446ux438ux435ux43dux442ux43eux432}

    \begin{Verbatim}[commandchars=\\\{\}]
{\color{incolor}In [{\color{incolor}23}]:} \PY{k}{def} \PY{n+nf}{find\PYZus{}center}\PY{p}{(}\PY{n}{img\PYZus{}coef\PYZus{}mostly\PYZus{}horizontal}\PY{p}{,} \PY{n}{img\PYZus{}coef\PYZus{}mostly\PYZus{}vertical}\PY{p}{)}\PY{p}{:}
             \PY{n}{img\PYZus{}coef\PYZus{}mostly\PYZus{}horizontal} \PY{o}{=} \PY{n}{img\PYZus{}coef\PYZus{}mostly\PYZus{}horizontal}\PY{o}{.}\PY{n}{astype}\PY{p}{(}\PY{l+s+s1}{\PYZsq{}}\PY{l+s+s1}{uint8}\PY{l+s+s1}{\PYZsq{}}\PY{p}{)}
             \PY{n}{y\PYZus{}max}\PY{p}{,} \PY{n}{x\PYZus{}max} \PY{o}{=} \PY{n}{img\PYZus{}coef\PYZus{}mostly\PYZus{}horizontal}\PY{o}{.}\PY{n}{shape}
             \PY{n}{width} \PY{o}{=} \PY{n}{y\PYZus{}max}\PY{o}{/}\PY{o}{/}\PY{l+m+mi}{2} \PY{c+c1}{\PYZsh{}width of original img}
             \PY{n}{height} \PY{o}{=} \PY{n}{x\PYZus{}max} \PY{o}{\PYZhy{}} \PY{n}{y\PYZus{}max} \PY{c+c1}{\PYZsh{}height of original img}
             \PY{n}{y\PYZus{}max}\PY{p}{,} \PY{n}{x\PYZus{}max} \PY{o}{=} \PY{n}{y\PYZus{}max}\PY{o}{\PYZhy{}}\PY{l+m+mi}{1}\PY{p}{,} \PY{n}{x\PYZus{}max}\PY{o}{\PYZhy{}}\PY{l+m+mi}{1} \PY{c+c1}{\PYZsh{}for calculate coord}
         
             \PY{n}{lines\PYZus{}H} \PY{o}{=} \PY{n}{cv}\PY{o}{.}\PY{n}{HoughLines}\PY{p}{(}\PY{n}{img\PYZus{}coef\PYZus{}mostly\PYZus{}horizontal}\PY{p}{,}\PY{l+m+mi}{1}\PY{p}{,}\PY{n}{np}\PY{o}{.}\PY{n}{pi}\PY{o}{/}\PY{l+m+mi}{180}\PY{p}{,}\PY{l+m+mi}{1}\PY{p}{,}\PY{n}{max\PYZus{}theta}\PY{o}{=}\PY{n}{pi}\PY{o}{/}\PY{l+m+mi}{2}\PY{o}{\PYZhy{}}\PY{l+m+mf}{1e\PYZhy{}2}\PY{p}{)} \PY{c+c1}{\PYZsh{}надо БПХ и аргмакс по хорошему}
             \PY{n}{rho}\PY{p}{,} \PY{n}{theta} \PY{o}{=} \PY{n}{lines\PYZus{}H}\PY{p}{[}\PY{l+m+mi}{0}\PY{p}{]}\PY{p}{[}\PY{l+m+mi}{0}\PY{p}{]}
             \PY{n}{a} \PY{o}{=} \PY{n}{np}\PY{o}{.}\PY{n}{cos}\PY{p}{(}\PY{n}{theta}\PY{p}{)}
             \PY{n}{b} \PY{o}{=} \PY{n}{np}\PY{o}{.}\PY{n}{sin}\PY{p}{(}\PY{n}{theta}\PY{p}{)}
             \PY{n}{x0} \PY{o}{=} \PY{n}{a}\PY{o}{*}\PY{n}{rho}
             \PY{n}{y0} \PY{o}{=} \PY{n}{b}\PY{o}{*}\PY{n}{rho}
             \PY{n}{tgalpha} \PY{o}{=} \PY{n}{tan}\PY{p}{(}\PY{n}{theta}\PY{o}{\PYZhy{}}\PY{n}{pi}\PY{o}{/}\PY{l+m+mi}{2}\PY{p}{)}
         
             \PY{n}{P3} \PY{o}{=} \PY{p}{(}\PY{n+nb}{int}\PY{p}{(}\PY{l+m+mi}{1}\PY{o}{/}\PY{n}{tgalpha}\PY{o}{*}\PY{p}{(}\PY{l+m+mi}{0}\PY{o}{\PYZhy{}}\PY{n}{y0}\PY{p}{)}\PY{o}{+}\PY{n}{x0}\PY{p}{)}\PY{p}{,} \PY{l+m+mi}{0}\PY{p}{)}
             \PY{n}{P4} \PY{o}{=} \PY{p}{(}\PY{n+nb}{int}\PY{p}{(}\PY{l+m+mi}{1}\PY{o}{/}\PY{n}{tgalpha}\PY{o}{*}\PY{p}{(}\PY{n}{y\PYZus{}max}\PY{o}{\PYZhy{}}\PY{n}{y0}\PY{p}{)}\PY{o}{+}\PY{n}{x0}\PY{p}{)}\PY{p}{,} \PY{n}{y\PYZus{}max}\PY{p}{)}
         
             \PY{n}{center\PYZus{}row} \PY{o}{=} \PY{p}{(}\PY{n}{P3}\PY{p}{[}\PY{l+m+mi}{0}\PY{p}{]}\PY{o}{+}\PY{n}{P4}\PY{p}{[}\PY{l+m+mi}{0}\PY{p}{]}\PY{o}{+}\PY{l+m+mi}{1}\PY{p}{)}\PY{o}{/}\PY{o}{/}\PY{l+m+mi}{2} \PY{o}{\PYZhy{}} \PY{n}{width}
         
             
             \PY{n}{img\PYZus{}coef\PYZus{}mostly\PYZus{}vertical} \PY{o}{=} \PY{n}{img\PYZus{}coef\PYZus{}mostly\PYZus{}vertical}\PY{o}{.}\PY{n}{astype}\PY{p}{(}\PY{l+s+s1}{\PYZsq{}}\PY{l+s+s1}{uint8}\PY{l+s+s1}{\PYZsq{}}\PY{p}{)}
             \PY{n}{y\PYZus{}max}\PY{p}{,} \PY{n}{x\PYZus{}max} \PY{o}{=} \PY{n}{img\PYZus{}coef\PYZus{}mostly\PYZus{}vertical}\PY{o}{.}\PY{n}{shape}
             \PY{n}{height} \PY{o}{=} \PY{n}{y\PYZus{}max}\PY{o}{/}\PY{o}{/}\PY{l+m+mi}{2} \PY{c+c1}{\PYZsh{}height of original img}
             \PY{n}{width} \PY{o}{=} \PY{n}{x\PYZus{}max} \PY{o}{\PYZhy{}} \PY{n}{y\PYZus{}max} \PY{c+c1}{\PYZsh{}width of original img}
             \PY{n}{y\PYZus{}max}\PY{p}{,} \PY{n}{x\PYZus{}max} \PY{o}{=} \PY{n}{y\PYZus{}max}\PY{o}{\PYZhy{}}\PY{l+m+mi}{1}\PY{p}{,} \PY{n}{x\PYZus{}max}\PY{o}{\PYZhy{}}\PY{l+m+mi}{1} \PY{c+c1}{\PYZsh{}for calculate coord}
         
             \PY{n}{lines\PYZus{}V} \PY{o}{=} \PY{n}{cv}\PY{o}{.}\PY{n}{HoughLines}\PY{p}{(}\PY{n}{img\PYZus{}coef\PYZus{}mostly\PYZus{}vertical}\PY{p}{,}\PY{l+m+mi}{1}\PY{p}{,}\PY{n}{np}\PY{o}{.}\PY{n}{pi}\PY{o}{/}\PY{l+m+mi}{180}\PY{p}{,}\PY{l+m+mi}{1}\PY{p}{,}\PY{n}{max\PYZus{}theta}\PY{o}{=}\PY{n}{pi}\PY{o}{/}\PY{l+m+mi}{2}\PY{o}{\PYZhy{}}\PY{l+m+mf}{1e\PYZhy{}2}\PY{p}{)} \PY{c+c1}{\PYZsh{}надо БПХ и аргмакс по хорошему}
             \PY{n}{rho}\PY{p}{,} \PY{n}{theta} \PY{o}{=} \PY{n}{lines\PYZus{}V}\PY{p}{[}\PY{l+m+mi}{0}\PY{p}{]}\PY{p}{[}\PY{l+m+mi}{0}\PY{p}{]}
             \PY{n}{a} \PY{o}{=} \PY{n}{np}\PY{o}{.}\PY{n}{cos}\PY{p}{(}\PY{n}{theta}\PY{p}{)}
             \PY{n}{b} \PY{o}{=} \PY{n}{np}\PY{o}{.}\PY{n}{sin}\PY{p}{(}\PY{n}{theta}\PY{p}{)}
             \PY{n}{x0} \PY{o}{=} \PY{n}{a}\PY{o}{*}\PY{n}{rho}
             \PY{n}{y0} \PY{o}{=} \PY{n}{b}\PY{o}{*}\PY{n}{rho}
             \PY{n}{tgalpha} \PY{o}{=} \PY{n}{tan}\PY{p}{(}\PY{n}{theta}\PY{o}{\PYZhy{}}\PY{n}{pi}\PY{o}{/}\PY{l+m+mi}{2}\PY{p}{)}
         
             \PY{n}{P3} \PY{o}{=} \PY{p}{(}\PY{n+nb}{int}\PY{p}{(}\PY{l+m+mi}{1}\PY{o}{/}\PY{n}{tgalpha}\PY{o}{*}\PY{p}{(}\PY{l+m+mi}{0}\PY{o}{\PYZhy{}}\PY{n}{y0}\PY{p}{)}\PY{o}{+}\PY{n}{x0}\PY{p}{)}\PY{p}{,} \PY{l+m+mi}{0}\PY{p}{)}
             \PY{n}{P4} \PY{o}{=} \PY{p}{(}\PY{n+nb}{int}\PY{p}{(}\PY{l+m+mi}{1}\PY{o}{/}\PY{n}{tgalpha}\PY{o}{*}\PY{p}{(}\PY{n}{y\PYZus{}max}\PY{o}{\PYZhy{}}\PY{n}{y0}\PY{p}{)}\PY{o}{+}\PY{n}{x0}\PY{p}{)}\PY{p}{,} \PY{n}{y\PYZus{}max}\PY{p}{)}
         
             \PY{n}{center\PYZus{}col} \PY{o}{=} \PY{p}{(}\PY{n}{P3}\PY{p}{[}\PY{l+m+mi}{0}\PY{p}{]}\PY{o}{+}\PY{n}{P4}\PY{p}{[}\PY{l+m+mi}{0}\PY{p}{]}\PY{o}{+}\PY{l+m+mi}{1}\PY{p}{)}\PY{o}{/}\PY{o}{/}\PY{l+m+mi}{2} \PY{o}{\PYZhy{}} \PY{n}{height}
             
             \PY{k}{return} \PY{n}{center\PYZus{}col}\PY{p}{,} \PY{n}{center\PYZus{}row}
\end{Verbatim}


    \begin{Verbatim}[commandchars=\\\{\}]
{\color{incolor}In [{\color{incolor}26}]:} \PY{o}{\PYZpc{}\PYZpc{}}\PY{k}{time}
         
         \PYZsh{}находим координаты центра объекта, используя пространство коэффициентов
         img\PYZus{}coef\PYZus{}H, img\PYZus{}coef\PYZus{}V = calculate\PYZus{}Hough(imgOrientation\PYZus{}GST)
         j, i = find\PYZus{}center(img\PYZus{}coef\PYZus{}H, img\PYZus{}coef\PYZus{}V)
\end{Verbatim}


    \begin{Verbatim}[commandchars=\\\{\}]
Wall time: 4.43 s

    \end{Verbatim}

    \begin{Verbatim}[commandchars=\\\{\}]
{\color{incolor}In [{\color{incolor}27}]:} \PY{c+c1}{\PYZsh{}center}
         \PY{n+nb}{print}\PY{p}{(}\PY{l+s+s1}{\PYZsq{}}\PY{l+s+s1}{размер изображения:}\PY{l+s+s1}{\PYZsq{}}\PY{p}{,} \PY{n}{img\PYZus{}stamp}\PY{o}{.}\PY{n}{shape}\PY{p}{)}
         \PY{n+nb}{print}\PY{p}{(}\PY{l+s+s1}{\PYZsq{}}\PY{l+s+s1}{центр объекта:}\PY{l+s+s1}{\PYZsq{}}\PY{p}{,} \PY{p}{(}\PY{n}{i}\PY{p}{,}\PY{n}{j}\PY{p}{)}\PY{p}{)}
\end{Verbatim}


    \begin{Verbatim}[commandchars=\\\{\}]
размер изображения: (439, 991)
центр объекта: (230, 350)

    \end{Verbatim}

    \begin{Verbatim}[commandchars=\\\{\}]
{\color{incolor}In [{\color{incolor}28}]:} \PY{c+c1}{\PYZsh{}найденный центр объекта на исходном изображении}
         \PY{n}{cv}\PY{o}{.}\PY{n}{circle}\PY{p}{(}\PY{n}{img\PYZus{}stamp\PYZus{}color}\PY{p}{,} \PY{p}{(}\PY{n}{j}\PY{p}{,}\PY{n}{i}\PY{p}{)}\PY{p}{,} \PY{l+m+mi}{10}\PY{p}{,} \PY{p}{(}\PY{l+m+mi}{255}\PY{p}{,}\PY{l+m+mi}{0}\PY{p}{,}\PY{l+m+mi}{0}\PY{p}{)}\PY{p}{,} \PY{l+m+mi}{2}\PY{p}{,} \PY{l+m+mi}{0}\PY{p}{)}
         \PY{n}{plt}\PY{o}{.}\PY{n}{imshow}\PY{p}{(}\PY{n}{img\PYZus{}stamp\PYZus{}color}\PY{p}{)}\PY{p}{;}
\end{Verbatim}


    \begin{center}
    \adjustimage{max size={0.9\linewidth}{0.9\paperheight}}{output_45_0.png}
    \end{center}
    { \hspace*{\fill} \\}
    

    % Add a bibliography block to the postdoc
    
    
    
    \end{document}
